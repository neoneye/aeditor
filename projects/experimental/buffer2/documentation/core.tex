\documentclass[a4paper,12pt]{article}
\author{Simon Strandgaard}
\title{\AE ditor 2.x Internals}
\begin{document}
\maketitle

\begin{abstract}
How is \AE ditor working internally. How is things implemented,
such as folding, syntax coloring, bookmarks, edit-strategies and undo. 
What decisions have been made. Talk about wide chars.
\end{abstract}

\section{Overview}
The core is divided into a \fbox{Model View Controller}.


\section{Model}
It maintains one long UTF-8 encoded string, which contains all
lines joined together. The benefits by having them in one string; 
that its quick to do take a snapshot of the entire text for undo/redo. 
That its fast to do replace. It allows searching for multiline patterns.

It maintains an array of integers, where each integer corresponds to 
the number of bytes a given line occupies. This makes its quick to 
do translations between position and byte-offsets.

The model has 3 important functions: p2b, b2p and replace.
Where only replace is the destructive function.
This narrow API helps ensuring that the models integrity stays good, 
no matter how hostile operations there are invoked.

Simple edit operations can be accomplished by using replace. For
instance the operation of inserting one letter at the cursor position,
is an erase nothing and insert a new letter. The operation of
issuing a backspace, is an erase the letter to be deleted and
insert nothing. More complex edit operations may require invoking 
replace multiple times.


\subsection{Invariants}
The string length must always be syncronized with the byte array.
\begin{equation}
\textrm{text.size} \quad = {\sum_{b \in \textrm{ bytes}} \!\!\! b}
\end{equation}

Text inserted must always be wellformed UTF-8, it must always
be inserte inbetween two other glyphs (which replaces ensures for us).

\subsection{Position to byte}
Converting from $x,y$ position to a byte offset.
$$
\textrm{findb}(b,n) = \cases{
  \textrm{findb}\bigg(\textrm{gbytes}(b) + b, n-1\bigg), 
    & for $n \geq 1$.\cr
  b & otherwise.}
$$
\begin{equation}
\textrm{p2b}(x,y) = \textrm{findb}\bigg(
  \sum_{i=0}^{y-1} \textrm{bytes}_i \quad, x\bigg)
\end{equation}
Where $\textrm{gbytes}(\textrm{byte})$ translates a byte 
index into a number of bytes the glyph occupies.
It relies on one of the nice things of UTF-8, that you can tell
how long a glyphs byte sequence are, just by looking at 
it's first byte. 
BTW: This is always true $ \textrm{p2b}(0,0) \rightarrow 0 $.

\subsection{Byte to position}
Converting from byte offset to a $x,y$ position.
$$
\textrm{findy}(b,y) = \cases{
  \textrm{findy}(b - \textrm{bytes}_y , y+1),
    & for $ b \geq \textrm{bytes}_y $.\cr
  y & otherwise.}
$$
$$
\textrm{findx}(b,n,x) = \cases{
  \textrm{findx}(b, n + \textrm{gbytes}(n), x+1),
    & for $ b \geq n + \textrm{gbytes}(n) $.\cr
  x & otherwise.}
$$
\begin{equation}
\textrm{b2p}(b) = \bigg< \textrm{findx}\bigg(b, 
  \sum_{i=0}^{\textrm{findy}(b,0)} \!\!\!\!\!\! \textrm{bytes}_i, 0\bigg);
  \quad \textrm{findy}(b,0) \bigg>
\end{equation}
BTW: This is always true $ \textrm{b2p}(0) \rightarrow <0,0> $.


\subsection{Replace}
The model is observable. Whenever a replacement operation is 
about to take place, then the observers are notified. This way
the views can stay in sync with the model.
\begin{enumerate}
\item notify observers before replacement.
\item replace area with new text
\item notify observers after replacement.
\end{enumerate}
The byte array is updated accordingly to the inserted text.


\newpage
\section{View}
In the future there probably will be more views. But right now
there exists only the edit view, which this section is about.

A view instance is attached to a model, whenever changes is made
to the model, then the view gets notification. The view can be 
attached to a canvas, on which the view can render glyphs.
The view can have attached a lexer, which deals with syntax coloring. 
Further more incoming events are being dispatched to the view. 

Lots of things is going on in the view. I never really like
having so many things going on at the same time. I like isolated and
encapsulated stuff. Its a mixed View / Mediator pattern which feels 
ackward, this is the only one of the GoF patterns that I never really 
have understood. Maybe I learn, maybe I don't.


\end{document}